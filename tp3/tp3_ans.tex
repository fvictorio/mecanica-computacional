\documentclass{article}

\usepackage[utf8]{inputenc}
\usepackage{enumerate}
\usepackage{graphicx}
\usepackage[spanish]{babel}
\usepackage{bm}
\usepackage{amsmath}%

\title{\textbf{Mecánica computacional - Trabajo Práctico 3}}
\author{FILARDI, Esteban; ROJAS FREDINI, Emmanuel; VICTORIO, Franco}
\date{}

\def\intone{\int_{-1}^1}
\def\intzeroone{\int_{0}^1}
\def\dx{\mbox{d}x}
\def\dy{\mbox{d}y}
\def\domega{\mbox{d}\Omega}
\def\dgamma{\mbox{d}\Gamma}
\def\summ{\sum_{m=1}^M}
\def\dxx#1{\frac{\mbox{d}^2 #1}{\mbox{d}x^2}}
\def\partialx#1{\frac{\partial #1}{\partial x}}
\def\partialy#1{\frac{\partial #1}{\partial y}}
\def\partialxx#1{\frac{\partial^2 #1}{\partial x^2}}
\def\partialyy#1{\frac{\partial^2 #1}{\partial y^2}}
\def\hatsigmax{\hat{\sigma}_x}
\def\hatsigmay{\hat{\sigma}_y}
\def\hattauxy{\hat{\tau}_{xy}}
\def\hatphi{\hat{\phi}}

\begin{document}

\maketitle

\begin{enumerate}[1)]
    \setcounter{enumi}{1}
    \item{ % 2)
        Si se divide uniformemente el intervalo en elementos y se desea utilizar funciones 
        cuadráticas, a cada elemento debe agregársele un nodo interno (a la mitad, por
        ejemplo) de modo de tener tres puntos para definir la parábola. Sean estos nodos
        $i$, $j$ y $k$ y sea $h$ el tamaño del elemento. Cada elemento tendrá entonces
        tres funciones de forma asociadas $N_\alpha$ que valen $1$ en el nodo $\alpha$
        y $0$ en los otros dos.

        Si se considera al elemento centrado en $0$, el nodo $i$ estará en $-\frac{h}{2}$,
        el nodo $j$ estará en $0$ y el nodo $k$ estará en $\frac{h}{2}$. La función
        $N_i(x)$ tendrá la forma $N_i(x) = a x^2 + b x + c$ y los coeficientes se
        obtienen haciendo $N_i(-\frac{h}{2}) = 1$, $N_i(0) = 0$ y $N_i(\frac{h}{2}) = 0$.
        Resolviendo el sistema, la función toma la forma $N_i(x) = \frac{2}{h^2} x^2
        - \frac{1}{h} x$. De forma similar se obtienen $N_j$ y $N_k$:

        \begin{eqnarray*}
            N_i(x) &=& \frac{2}{h^2} x^2 - \frac{1}{h} x \\
            N_j(x) &=& -\frac{4}{h^2} x^2 + 1 \\
            N_k(x) &=& \frac{2}{h^2} x^2 + \frac{1}{h} x \\
        \end{eqnarray*}

        Se tiene el problema $-\partialxx{\phi} = f(x)$ y se utiliza una aproximación
        a la solución de la forma $\hat{\phi} = \summ \phi_m N_m$, el residuo en el
        interior es $R_\Omega = f(x) + \partialxx{\hatphi}$. Utilizamos el método
        de residuos ponderados y obtenemos:

        \begin{eqnarray*}
            \int_\Omega W_l R_\Omega \domega &=& 0 \\
            \intzeroone W_l \left( f(x) + \partialxx{\hatphi} \right) \dx &=& 0 \\
        \end{eqnarray*}

        Debilitando en la última expresión, ésta se transforma en:

        \begin{eqnarray*}
            \intzeroone W_l f(x) \dx - 
            \intzeroone \partialx{W_l} \partialx{\hatphi} \dx + 
            \left[ W_l \partialx{\hatphi} \right]_0^1  &=& 0 \\
            \intzeroone \partialx{W_l} \partialx{\hatphi} \dx -
            \left[ W_l \partialx{\hatphi} \right]_0^1 &=& 
            \intzeroone W_l f(x) \dx \\
            \summ \phi_m \intzeroone \partialx{N_l} \partialx{N_m} \dx -
            \left[ N_l \summ \phi_m \partialx{N_m} \right]_0^1 &=& \intzeroone N_l f(x) \dx  \\
        \end{eqnarray*}

        Como todas las funciones de forma salvo la primera y la última son cero en los extremos,
        y como $\phi_m$ se anula en estos puntos, la expresión de contorno es cero y la expresión
        resultante es:

        \begin{eqnarray*}
            \summ \phi_m \intzeroone \partialx{N_l} \partialx{N_m} \dx
            &=& \intzeroone N_l f(x) \dx  \\
        \end{eqnarray*}

        Esto puede escribirse en la forma $\boldsymbol{K} \boldsymbol{\phi} = \boldsymbol{f}$,
        donde:

        \[ K_{lm} = \intzeroone \partialx{N_l} \partialx{N_m} \dx \]
        \[ f_l = \intzeroone N_l f(x) \dx \]

        Como las funciones de forma utilizadas son de soporte compacto, la matriz
        $\boldsymbol{K}$ puede expresarse como el ensamblado de $M-1$ matrices
        elementales $\boldsymbol{K^e}$ (una por cada elemento). En cada elemento
        hay tres funciones de forma no-nulas, y por lo tanto las matrices
        $\boldsymbol{K^e}$ serán de tres por tres. Como se asume que el intervalo
        está definido uniformemente, todas las $\boldsymbol{K^e}$ tendrán la misma
        forma. El componente $K_{1 1}^e$ por ejemplo se obtiene haciendo:

        \[ \int_{-\frac{h}{2}}^{\frac{h}{2}} \left( \partialx{N_i} \right)^2 \dx 
           = \frac{7}{3h} \]

        Y el componente $K_{1 2}^e$:

        \[ \int_{-\frac{h}{2}}^{\frac{h}{2}} 
        \left( \partialx{N_i} \partialx{N_j} \right) \dx 
           = -\frac{8}{3h} \]

        De la misma forma se obtiene el resto de los componentes. La matriz elemental
        resultante es:

        \[ \boldsymbol{K^e} = \frac{1}{3 h} \left( \begin{array}{ccc} 
        7 & -8 & 1 \\
        -8 & 16 & -8 \\
        1 & -8 & 7 \\
        \end{array} \right) \]

        La matriz $\boldsymbol{K}$ se obtiene ensamblando estas matrices elementales.
        Por ejemplo, si se tienen tres elementos:

        \[ \boldsymbol{K} = \frac{1}{3h} \left( \begin{array}{ccccccc}
        7 & -8 & 1 & 0 & 0 & 0 & 0       \\
        -8 & 16 & -8 & 0 & 0 & 0 & 0     \\
        1 & -8 & (7+7) & -8 & 1 & 0 & 0  \\
        0 & 0  & -8 & 16 & -8 & 0 & 0    \\
        0 & 0  & 1 & -8 & (7+7) & -8 & 1 \\
        0 & 0  & 0 & 0  & -8 & 16 & -8   \\
        0 & 0  & 0 & 0  & 1  & -8 & 7    \\
        \end{array} \right) \]
    }
    \setcounter{enumi}{3}
    \item{ % 4)
        Ecuacion Diferencial:

        \begin{equation}
            \frac{\partial }{\partial x}\left(k \frac{\partial T}{\partial x}\right) + \frac{\partial }{\partial y} \left( k \frac{\partial T}{\partial y} \right) + Q - c \left( T - T_{amb} \right) = 0
        \end{equation}

        \begin{itemize}
            \item Donde k cte escalar.
            \item Donde Q cte escalar, al menos dentro de 1 elemento.
        \end{itemize}

        CC:
        \begin{align}
            k \frac{\partial T}{\partial \vec{n}} &= -h\left( T- T_{amb} \right)   &\Gamma_{\inf}\\
            k \frac{\partial T}{\partial \vec{n}} &= -\bar{q}   &\Gamma_{q}\\
            T                                     &=  \bar{T} &\Gamma_{\phi}
        \end{align}

        Nuestra aproximacion sera:

        \begin{equation}
            T \approx \hat{T} = \sum{ a_m N_m\left(x\right)}
        \end{equation}

        Las condiciones Dirichlet en los bordes $\Gamma_{\phi}$ se impondran en la matriz ensamblada, es decir se impondran los 
        coeficientes $a$ a el valor correspondiente.\\

        Nuestra aproximacion de MRP sera:

        \begin{align*}
            \int{W_l \left( k \frac{\partial^2 \hat{T}}{\partial x^2} + Q + c\left(\hat{T}-T_{amb}\right) \right) d\Omega} +\\
             \int{ \bar{W_l} \left( -k\frac{\partial \hat{T}}{\partial \vec{n}} + h\left(\hat{T}-T_{amb}\right) \right) d\Gamma_{\inf}} + \int{\bar{W_l} \left(k\frac{\partial \hat{T}}{\partial \vec{n}}+ \bar{q} \right) d\Gamma_{q}}= 0
        \end{align*}

        Debilitamos:

        \begin{align*}
            - k\int{ \frac{\partial W_l}{\partial x} \frac{\partial \hat{T}}{\partial x} d\Omega} + k\int{ W_l \frac{\partial \hat{T}}{\partial \vec{n}} d\Gamma_{\inf+q+\phi}} + \int{W_l \left( Q + c\left(\hat{T}-T_{amb}\right) \right) d\Omega} +\\
            \int{ \bar{W_l} \left( -k\frac{\partial \hat{T}}{\partial \vec{n}} + h\left(\hat{T}-T_{amb}\right) \right) d\Gamma_{\inf}} + \int{\bar{W_l} \left(k\frac{\partial \hat{T}}{\partial \vec{n}}+ \bar{q} \right) d\Gamma_{q}} = 0
        \end{align*}

        Elegimos: 
        \begin{align*}
            \bar{W_l} &= -W_l \hspace{10pt} &en \hspace{10pt} \Gamma_{\inf}\\
            W_l       &= 0	  \hspace{10pt} &en \hspace{10pt} \Gamma_{\phi}\\
            \bar{W_l} &= -W_l \hspace{10pt} &en \hspace{10pt} \Gamma_{q}\\
        \end{align*}

        \begin{equation*}
            - k\int{ \frac{\partial W_l}{\partial x} \frac{\partial \hat{T}}{\partial x} d\Omega} + \int{W_l \left( Q + c\left(\hat{T}-T_{amb}\right) \right) d\Omega} - h \int{ W_l \left(\hat{T}-T_{amb}\right) d\Gamma_{\inf}} - \int{ W_l \bar{q} d\Gamma_{q}} = 0
        \end{equation*}

        Usamos Galerkin, es decir $W_l = N_l$:

        \begin{equation*}
            - k\int{ \frac{\partial N_l}{\partial x} \frac{\partial \hat{T}}{\partial x} d\Omega} + \int{N_l \left( Q + c\left(\hat{T}-T_{amb}\right) \right) d\Omega} - h \int{ N_l \left(\hat{T}-T_{amb}\right) d\Gamma_{\inf}} - \int{ N_l \bar{q} d\Gamma_{q}} = 0
        \end{equation*}

        Reemplazamos $\hat{T}$:

        \begin{align*}
            \sum{ - k\int{ \frac{\partial N_l}{\partial x} \frac{\partial \sum{a_m N_m}}{\partial x} d\Omega} + \int{N_l \left( Q + c\left(\sum{a_m N_m}-T_{amb}\right) \right) d\Omega} - }\\
            h \int{ N_l \left(\sum{a_m N_m}-T_{amb}\right) d\Gamma_{\inf}} - \int{ N_l \bar{q} d\Gamma_{q}} = 0
        \end{align*}

        \begin{align*}
            \sum{ - k\int{ \frac{\partial N_l}{\partial x} \frac{\partial N_m}{\partial x} d\Omega} a_m + c\int{N_l N_m d\Omega} a_m + \int{N_l \left( Q - cT_{amb}\right) d\Omega} - }\\ 
            h \int{ N_l \left(N_m-T_{amb}\right) d\Gamma_{\inf}} a_m - \int{ N_l \bar{q} d\Gamma_{q}} = 0
        \end{align*}

        \begin{align*}
            \sum{ \left[ - k\int{ \frac{\partial N_l}{\partial x} \frac{\partial N_m}{\partial x} + c N_l N_m  d\Omega} - h \int{ N_l \left(N_m-T_{amb}\right) d\Gamma_{\inf}} \right] a_m }  &=\\
            &-\int{N_l \left( Q - cT_{amb}\right) d\Omega} + \int{ N_l \bar{q} d\Gamma_{q}}
        \end{align*}

        Las matrices elementales son:

        \begin{align}
            K_{l,m}^{e} &= k\int{ \frac{\partial N_l}{\partial x} \frac{\partial N_m}{\partial x} + c N_l N_m  d\Omega^{e}} - h \int{ N_l \left(N_m-T_{amb}\right) d\Gamma_{\inf}^{e}}\\
            f_{l}^{e}   &= \int{N_l \left( Q - cT_{amb}\right) d\Omega^{e}} - \int{ N_l \bar{q} d\Gamma_{q}}
        \end{align}

        \begin{enumerate}[I.]
        \item \textbf{Caso particular}

        Se condidera el caso donde:
        \begin{itemize}
            \item c = 0
            \item k = 1
            \item Q = 1 para $x \in \left[0,\frac{1}{2}\right]$
            \item Q = 0 para $x \in \left(\frac{1}{2},1\right]$
        \end{itemize}

        Condiciones de contorno:
        \begin{itemize}
            \item $\bar{T} = 1$ en $x=0$
            \item $\bar{T} = 0$ en $x=1$
        \end{itemize}

        Ya que las condiciones Dirichlet las impondremos en la matriz ensamblada luego las matrices elementales se simplificaran a:

        \begin{align}
            K_{l,m}^{e} &=  k\int{ \frac{\partial N_l}{\partial x} \frac{\partial N_m}{\partial x} d\Omega^{e}}\\
            f_{l}^{e}   &= Q \int{N_l d\Omega^{e}}
        \end{align}

        \item \textbf{Funciones de Forma}

        Usaremos elementos formados por 2 nodos y funciones de forma lineales que sera:

        \begin{itemize}
            \item $N_i = \frac{h-x}{h}$
            \item $N_j = 1-\frac{h-x}{h}$
        \end{itemize}

        Donde $h$ es el ancho de cada elemento.
        Con derivadas:

        \begin{itemize}
            \item $\frac{\partial N_i}{\partial x} = \frac{-1}{h}$
            \item $\frac{\partial N_j}{\partial x} = \frac{1}{h}$
        \end{itemize}

        \item \textbf{Resultados}

        Tomaremos $L=1$. El resultado con 2 condiciones de contorno Dirichlet siguientes es:

        \begin{itemize}
            \item En $x=0$ condicion Dirichlet $T=1$
            \item En $x=1$ condicion Dirichlet $T=0$
        \end{itemize}

        \begin{center}
            \includegraphics[scale=0.8]{Ejercicio2Dirichlet.png}
        \end{center}

        El resultado con 1 condiciones de contorno Dirichlet y una Newmann tal que:

        \begin{itemize}
            \item En $x=0$ condicion Dirichlet $T=1$
            \item En $x=1$ condicion Newmann   $k\frac{\partial T}{\partial \vec{n}}=0$
        \end{itemize}

        \begin{center}
            \includegraphics[scale=0.8]{Ejercicio2Newmann.png}
        \end{center}

        Haciendo pruebas de error de la aproximacion del metodo de elementos finitos, en el caso de fronteras Dirchlet, con respecto
        a la solucion analitica, podemos ver que a medida que refinamos la malla el error disminuye de la siguiente forma:

        \begin{center}
            \includegraphics[scale=0.8]{ErrorVsCantidadNodos.png}
        \end{center}

        El error se midio como el error cuadratico medio.\\
        Es decir que el error disminuye de forma cuadratica a medida que se agregan nodos a la malla, y este aumento de nodos en la malla
        significa un aumento igual de elementos, ya que se tomaron elementos lineales formados por 2 nodos.\\
        Podemos ver que la grafica de la funcion $y=\frac{1}{x^2}$ es:

        \begin{center}
            \includegraphics[scale=0.8]{Y=1DivXCuadrado.png}
        \end{center}

        Tiene la misma tendencia que el resultado del error por esto decimos que el error disminuye cuadraticamente con la cantidad de nodos.
        \end{enumerate}
    }
    \item{ % 5)
        \begin{enumerate}[a)]
        \item{
            Utilizando $L = b(x) = P = E = A = 1$, los desplazamientos y tensiones son:
            \includegraphics[width=\textwidth]{5ad.png}
            \includegraphics[width=\textwidth]{5at.png}
        }
        \item{
            Utilizando $L = b(x) = P = E = 1$, los desplazamientos y tensiones son:
            \includegraphics[width=\textwidth]{5bd.png}
            \includegraphics[width=\textwidth]{5bt.png}

            (En ningún caso se grafican las deformaciones ya que son iguales a las tensiones
            por ser $E=1$.)
        }
        \end{enumerate}
    }
\end{enumerate}
\end{document}

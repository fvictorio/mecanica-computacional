\documentclass{article}

\usepackage[utf8]{inputenc}
\usepackage{enumerate}
\usepackage{graphicx}
\usepackage[spanish]{babel}
\usepackage{bm}
\usepackage{amsmath}
\usepackage{hyperref}

\title{\textbf{Mecánica computacional - Trabajo Práctico 4}}
\author{FILARDI, Esteban; ROJAS FREDINI, Emmanuel; VICTORIO, Franco}
\date{}

\def\intone{\int_{-1}^1}
\def\intzeroone{\int_{0}^1}
\def\dx{\mbox{d}x}
\def\dy{\mbox{d}y}
\def\domega{\mbox{d}\Omega}
\def\domegae{\mbox{d}\Omega^e}
\def\dgamma{\mbox{d}\Gamma}
\def\intomega#1{\int_\Omega #1 \domega}
\def\intomegae#1{\int_{\Omega^e} #1 \domega}
\def\summ{\sum_{m=1}^M}
\def\dxx#1{\frac{\mbox{d}^2 #1}{\mbox{d}x^2}}
\def\partialx#1{\frac{\partial #1}{\partial x}}
\def\partialy#1{\frac{\partial #1}{\partial y}}
\def\partialn#1{\frac{\partial #1}{\partial n}}
\def\partialxx#1{\frac{\partial^2 #1}{\partial x^2}}
\def\partialyy#1{\frac{\partial^2 #1}{\partial y^2}}
\def\hatsigmax{\hat{\sigma}_x}
\def\hatsigmay{\hat{\sigma}_y}
\def\hattauxy{\hat{\tau}_{xy}}
\def\hatphi{\hat{\phi}}

\begin{document}

\maketitle

\begin{enumerate}[1)]
    \item{ % 1)
        Para una variable dependiente $T$ el operador laplaciano toma la forma
        $\Delta T = \partialxx{T} + \partialyy{T}$. Si éste forma parte de la ecuación
        del método de residuos ponderados con Galerkin, el término correspondiente es:

        \[ \intomega{N_l \left( \partialxx{T} + \partialyy{T} \right)} \]

        Si se debilita para evitar la segunda derivada (e ignorando el término
        en la frontera) se obtiene:

        \[ \intomega{\left(\partialx{N_l}\partialx{T} + \partialy{N_l}\partialy{T}\right)} \]

        Si se expresa $T$ como $T = \summ T_m N_m$ y se utiliza el método de
        elementos finitos, la matriz elemental de cada elemento correspondiente al operador
        laplaciano tiene la forma:

        \[ K_{lm}^e = \intomegae{\left( \partialx{N_l}\partialx{N_m} + \partialy{N_l}\partialy{N_m}\right)} \]

        Si se está trabajando con elementos triangulares genéricos, las funciones de forma $N_i$
        se expresan como:

        \[ N_i = \frac{a_i + b_i x + c_i y}{2A} \]

        Donde $A$ es el área del elemento y los términos $a_i$, $b_i$ y $c_i$ se calculan
        con las coordenadas de los nodos. Las derivadas parciales son entonces:

        \begin{eqnarray*}
            \partialx{N_i} &=& \frac{b_i}{2A} \\
            \partialy{N_i} &=& \frac{c_i}{2A}
        \end{eqnarray*}

        La matriz elemental toma entonces la forma

        \begin{eqnarray*}
            K_{lm}^e &=& \intomegae{\left( \partialx{N_l}\partialx{N_m} + \partialy{N_l}\partialy{N_m}\right)} \\
            &=& \intomegae{\frac{b_l b_m + c_l c_m}{4 A^2}} \\
            &=& \frac{b_l b_m + c_l c_m}{4 A^2} \intomegae{} \\
            &=& \frac{b_l b_m + c_l c_m}{4 A^2} A \\
            &=& \frac{b_l b_m + c_l c_m}{4 A} \\
        \end{eqnarray*}

        Si en cambio se utilizan elementos cuadrangulares bilineales alineados con los ejes,
        las funciones de forma son:

        \begin{eqnarray*}
            N_i &=& \frac{(x_k-x)(y_k-y)}{A} \\
            N_j &=& \frac{(x-x_i)(y_k-y)}{A} \\
            N_k &=& \frac{(x-x_i)(y-y_i)}{A} \\
            N_l &=& \frac{(x_k-x)(y-y_i)}{A} \\
        \end{eqnarray*}

        Donde el nodo $i$ corresponde a la esquina inferior izquierda y el resto se numera
        en sentido antihorario, $x_i$ e $y_i$ son las coordenadas del nodo $i$ y
        $x_k$ e $y_k$ son las coordenadas del nodo $k$.

        Las derivadas parciales correspondientes son:

        \[
            \begin{array}{llll}
                \partialx{N_i} = \frac{y_k-y}{A} & 
                \partialx{N_j} = \frac{y_k-y}{A} & 
                \partialx{N_k} = \frac{y-y_y}{A} & 
                \partialx{N_l} = \frac{y-y_y}{A} \\[0.25cm]
                \partialy{N_i} = \frac{x_k-x}{A} & 
                \partialy{N_j} = \frac{x-x_i}{A} & 
                \partialy{N_k} = \frac{x-x_i}{A} & 
                \partialy{N_l} = \frac{x_k-x}{A} \\
            \end{array}
        \]
        
        Usando estas derivadas en $K_{lm}^e = \intomegae{\left( \partialx{N_l}\partialx{N_m} + 
        \partialy{N_l}\partialy{N_m}\right)}$ se obtiene la matriz elemental deseada.
        Nótese que, a diferencia de los elementos triangulares, los integrandos no son
        constantes y por lo tanto se debe integrar para cada componente de la matriz.
    }
    \item{ % 2)
        El residuo en el interior del dominio es:

        \[ R_\Omega = k\left(\partialxx{T} + \partialyy{T}\right) + Q(x, y) \]

        mientras que en los contornos con condiciones Neumann y Robin es, respectivamente:

        \begin{eqnarray*}
        R_{\Gamma_q} &=& k \partialn{T} + \bar{q} \\
        R_{\Gamma_h} &=& h\left(T-T_\infty\right) + k \partialn{T}
        \end{eqnarray*}

        En el problema dado, la normal apunta hacia la dirección negativa de las $x$ en el
        caso Neumann y para la dirección positiva de las $y$ en el Robin, por lo que podemos
        reescribir esto como:

        \begin{eqnarray*}
        R_{\Gamma_q} &=& -k \partialx{T} + \bar{q} \\
        R_{\Gamma_h} &=& h\left(T-T_\infty\right) + k \partialy{T}
        \end{eqnarray*}

        Utilizando estos residuos en el método de residuos ponderados, usando Galerkin y
        aplicando el método de elementos finitos y debilitando, se llega a la siguiente ecuación
        (donde no se colocan las sumatorias en $T_m$ por comodidad):

        \begin{eqnarray*}
            &&\intomegae{k \left[ \partialx{N_l}\partialx{N_m} + \partialy{N_l}\partialy{N_m}\right]} 
            - \int_{\Gamma_h^e} \frac{h}{k} N_l N_m \dgamma
            =\\
            &&\intomegae{N_l Q(x,y)} + \int_{\Gamma_q^e} \frac{\bar{q}}{k}\dgamma
            - \int_{\Gamma_h^e} \frac{h T_\infty}{k} \dgamma
        \end{eqnarray*}
        
        A continuación se muestra el resultado obtenido dividiendo la placa en ocho elementos:

        \includegraphics[width=\textwidth]{ej2.png}
    }
    \item{ % 3)
        En \href{http://youtu.be/ZxNBwtYpoPg}{este video} puede observarse cómo se distribuye la temperatura
        en el cuerpo. (Ignórese el cuadrado correspondiente al agujero interior y los triángulos
        arriba a la derecha y abajo a la derecha, que aparecen por graficar usando la triangulación
        Delaunay.) Para obtener este resultado se utilizó un esquema temporal explícito con un paso
        de tiempo de $0.01$. El mallado utilizado no corresponde exactamente al dado en el ejercicio si no
        a uno obtenido con GID.
    }
\end{enumerate}
\end{document}
